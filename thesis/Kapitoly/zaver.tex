V~této práci byly popsány existující nástroje pro ladění, profilování a~instrumentaci CUDA programů, obecné principy profilování a~technologie CUDA a~LLVM. Pro účely profilování aplikací na grafické kartě byl vytvořen instrumentační nástroj, který umožňuje vložit do CUDA programů při překladu dodatečné instrukce. Ty pak za běhu programu zaznamenávají paměťové operace na grafické kartě. Zaznamenané údaje obsahují detailní informace o~vláknu provádějícím přístup i~o~samotném přístupu. Kromě toho také nástroj zaznamenává veškeré paměťové alokace na grafické kartě. Veškeré zaznamenané údaje jsou poté uloženy do souborů ve formátech JSON, Protobuf nebo Cap'n Proto. Otestováním nástroje bylo zjištěno, že sice zpomaluje dobu výpočtu, ale ne natolik, aby to omezovalo jeho použití. Nástroj je jednoduché použít k~instrumentaci existujícího kódu, stačí přidat několik přepínačů překladači a~do zdrojového souboru s~kernelem vložit jeden hlavičkový soubor. Vytvořený profilovací nástroj tak splňuje požadavky, které na něho byly kladeny. Zároveň dokazuje, že pomocí překladače Clang lze jednoduše instrumentovat programy na grafických kartách. Nástroj byl během svého vývoje upraven pro podporu dvou nových verzí překladače Clang (5 a~6) a~tyto úpravy nevyžadovaly velké změny v~kódu. Díky použití frameworku LLVM tak je jednodušší psát udržitelné instrumentační nástroje. Právě špatná udržitelnost způsobená uzavřeností platformy CUDA způsobuje rychlé zastarávání ostatních nástrojů pro instrumentaci CUDA kódu.

Dále byla v~práci vytvořena webová aplikace, která soubory vygenerované instrumentovaným programem načítá a~vizualizuje jejich obsah. Pomocí této aplikace lze zobrazit a~vyhledávat jednotlivé warpy a~vlákna, které prováděly paměťové přístupy. Pro zvolené warpy pak lze zobrazit paměťové konflikty přístupů k~modulům sdílené paměti a~přístupům ke globální paměti. Dále lze v~aplikaci pro každý paměťový přístup zobrazit jeho umístění v~adresním prostoru aplikace a~sledovat tak vzory přístupů sousedních vláken ve warpu. Tento pohled na paměťové přístupy usnadňuje intuitivní představu o~tom, jak vlákna pracují s~pamětí a~může tak programátorovi pomoct tyto přístupy optimalizovat.

Instrumentační nástroj obsahuje funkční základ pro modifikaci CUDA programů a~je snadné k~němu přidat nové instrumentační funkce.
Nástroj by šel rozvíjet do šířky, například přidáním podpory pro více CUDA alokačních funkcí nebo instrumentací konkrétních PTX instrukcí, které by poté šlo vizualizovat. Alternativou k~současnému způsobu instrumentace by mohla být instrumentace zdrojového kódu, která by odstranila závislost nástroje na překladači Clang. Ta by však nemohla zachytit detaily o~nízkoúrovňových detailech programu.
Nástroj by mohl být rozšířen o~zaznamenávání paměťových přístupů CPU. To by vyžadovalo drobné změny formátu generovaných souborů a~nový pohled na přístupy ve webové aplikaci. Zaznamenání přístupů na procesoru by šlo provést například pomocí již zmíněných nástrojů Pin\cite{pin} nebo Valgrind\cite{valgrind}. Nástroj by poté sloužil jako univerzání vizualizátor paměťových přístupů jak na procesoru, tak na grafické kartě. Vizualizační aplikace by mohla být rozšířena o~podporu automatické detekce problémových situací, jako jsou například konflikty při přístupu do sdílené paměti nebo vzdálené přístupy sousedních vláken do paměti. Tato detekce by usnadnila hledání výkonnostních problémů, které teď musí uživatel provádět manuálně.

Instrumentace programů může být velmi užitečným nástrojem pro profilování programů. Uzavřenost platformy CUDA bohužel komplikuje tvorbu nástrojů pro grafické karty od Nvidie. S~rozvíjející se podporou CUDY v~překladači Clang se to snad v~budoucnu zlepší. Díky této platformě můžou vznikat nové profilovací nástroje, které usnadní optimalizaci paralelních programů pro grafické karty a~které půjde snadno udržovat.
