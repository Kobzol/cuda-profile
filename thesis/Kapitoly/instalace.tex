\subsection{Profilovací nástroj}
Profilovací nástroj byl primárně vyvíjen pro Linuxové distribuce založené na Debianu, instalační instrukce jsou tedy zaměřené na ně. Jeho kód se nachází ve složce \texttt{collection}, následující příkazy předpokládají spouštění z~této složky. 

\subsubsection{Závislosti}
Projekt lze sestavit pomocí programu \emph{CMake}\footnote{\url{https://cmake.org}}, který ze zdrojového kódu generuje soubory pro konkrétní sestavovací systém (např. \emph{make}). Před sestavením projektu je nutné nainstalovat závislosti CMake a~LLVM. Instrumentační nástroj by měl být kompatibilní s~LLVM 4, 5 i~6 a~s~CUDOU ve verzi 8 a~9. Podpora pro CUDA 9.1 je v~době tvorby této práce stále ve vývoji.
Mezi verzí CUDY 8 a~9 vznikly relativně velké změny ve spolupráci CUDY s~Clangem, doporučená a~nejvíce otestovaná kombinace verzí je pro použití profilovacího nástroje je LLVM 4 a~CUDA 8.
\begin{minted}{bash}
  $ apt-get install build-essential cmake llvm-4.0-dev clang-4.0
\end{minted}
Pro použití komprese a~knihoven Protobuf či Cap'n Proto je potřeba doinstalovat tyto balíčky. Aplikace používá Protobuf ve verzi 2 a~Cap'n Proto ve verzi 0.6.
\begin{minted}{bash}
  $ apt-get install libprotobuf-dev protobuf-compiler capnproto libcapnp-dev zlib1g-dev
\end{minted}
Testy jsou napsány v~Pythonu 2.7 a~jsou určené pro testovací knihovnu \emph{pytest}\footnote{\url{https://docs.pytest.org/en/latest}}. Pro spuštění testů je nutné doinstalovat balíčky s~Pythonem, pytestem a~dále nainstalovat Protobuf a~Cap'n Proto pro Python.
\begin{minted}{bash}
  $ apt-get install python python-pip python-pytest
  $ pip install protobuf pycapnp
\end{minted}

\subsubsection{Sestavení}
Pro sestavení projektu do složky \emph{build} lze použít tyto příkazy:

\begin{minted}{bash}
  $ mkdir build
  $ cd build
  $ cmake -DCMAKE_BUILD_TYPE=RELEASE ..
  $ make -j
\end{minted}
Po sestavení lze spustit testy pomocí příkazu
\begin{minted}{bash}
  $ py.test tests
\end{minted}
v~kořenovém adresáři projektu.

Pro zjednodušení instalace profilovacího nástroje je v~repozitáři k~dispozici konfigurační soubor pro \emph{Docker}\footnote{\url{https://www.docker.com}}, který obsahuje všechny potřebné závislosti. Návod pro instalaci Dockeru lze najít například zde: \url{https://docs.docker.com/install/linux/docker-ce/ubuntu}.
Jelikož profilovací nástroj vyžaduje CUDA grafickou kartu pro své spuštění, musí se Docker kontejner spustit s~podporou běhového prostředí CUDA. Pro sestavení projektu v~Dockeru a~následné spuštění testů lze použít tyto příkazy v~kořenovém adresáři projektu:
\begin{minted}{bash}
  $ apt-get install nvidia-docker2
  $ docker build -t profiler .
  $ docker run --runtime=nvidia profiler
\end{minted}

Kontejner je možné zapnout v~interaktivním režimu a~použít jej k~instrumentaci a~spuštění CUDA programů.

\subsubsection{Použití}
\label{manual:usage}
Pro každý soubor s~kernely, které se mají instrumentovat, je nejprve nutné vložit do zdrojového \texttt{.cu} souboru hlavičkový soubor \texttt{CuprRuntime.h} ze složky \texttt{include}.
\begin{minted}{cpp}
  #include <CuprRuntime.h>
\end{minted}
Aby šlo takto soubor vložit, tak musí být složka \texttt{include} přidána do seznamu cest ke hlavičkovým souborům v~překladači.\footnote{Nestačí zkopírovat hlavičkový soubor do jiného projektu, protože se odkazuje na jiné soubory profilovacího nástroje pomocí relativních cest}

Instrumentovaný program se poté musí přeložit pomocí překladače Clang. Při překladu je nutné přilinkovat knihovny profilovacího nástroje. Následující skript přeloží zdrojové soubory a~provede instrumentaci profilovacím nástrojem. Skript předpokládá, že v~proměnné prostředí CUPR\_BUILD\_DIR je absolutní cesta k~adresáři se sestaveným profilovacím nástrojem (tedy např. adresář \texttt{build}, viz výše).

\begin{minted}[escapeinside=\#\#]{bash}
$ clang++ -std=c++14 --cuda-gpu-arch=sm_30 \
    -I/usr/local/cuda/include -L/usr/local/cuda/lib64 \
    #\colorbox{insthighlight}{-I\$\{CUPR\_BUILD\_DIR\}/include}# \
    #\colorbox{insthighlight}{-L\$\{CUPR\_BUILD\_DIR\}/runtime}# \
    #\colorbox{insthighlight}{-Xclang -load -Xclang \$\{CUPR\_BUILD\_DIR\}/instrument/libinstrument.so}# \
    #\colorbox{insthighlight}{-lruntime}# \
    #\colorbox{insthighlight}{-z muldefs}# \
    -lcudart \
    -xcuda \
    <zdrojové soubory> -o cuda
\end{minted}
Zvýrazněné řádky jsou potřeba pro instrumentaci, ostatní přepínače jsou standardní při překládání CUDA programů překladačem Clang. Přepínač \texttt{-z\ muldefs} je nutný pouze pokud je soubor \texttt{CuprRuntime.h} vložen do více než jednoho souboru s~kernely. Tento přepínač ignoruje výskyt více definic stejné deklarace a~je nutný z~důvodu omezení Clangu popsaného v~sekci \ref{sec:gpucollection}.

Níže je pro úplnost ukázka minimálního souboru pro instrumentaci a~sestavení CUDA programu překladačem Clang pomocí CMake.


\begin{minted}[fontsize=\small]{cmake}
cmake_minimum_required(VERSION 3.4)

set(CMAKE_CXX_STANDARD 14)
set(CMAKE_CXX_EXTENSIONS OFF)

set(CUPR_SRC_DIR <repozitář projektu>)
set(CUPR_BUILD_DIR <složka se sestaveným projektem>)

find_package(CUDA REQUIRED)

set(CMAKE_CXX_COMPILER clang++)
set(CMAKE_CXX_FLAGS "${CMAKE_CXX_FLAGS} -g --cuda-gpu-arch=sm_30 -Xclang -load \
-Xclang ${CUPR_BUILD_DIR}/instrument/libinstrument.so")

set_source_files_properties(kernel.cu PROPERTIES LANGUAGE CXX)

include_directories(${CUDA_INCLUDE_DIRS})
link_directories("${CUDA_TOOLKIT_ROOT_DIR}/lib64")
link_directories("${CUPR_BUILD_DIR}/runtime")

add_executable(profiled main.cpp kernel.cu)
target_link_libraries(profiled cudart runtime)
target_include_directories(profiled PRIVATE ${CUPR_SRC_DIR}/include)
\end{minted}


Instrumentovaný program potřebuje k~běhu dynamickou knihovnu profilovacího nástroje, je tedy nutné buď vytvořit symbolický odkaz ke knihovně pomocí programu \texttt{ldconfig}\footnote{\url{http://man7.org/linux/man-pages/man8/ldconfig.8.html}} nebo použít proměnnou prostředí LD\_LIBRARY\_PATH\footnote{\url{http://man7.org/linux/man-pages/man8/ld.so.8.html}} pro určení umístění knihovny. Následuje kompletní ukázka spuštění programu i~s~použitím knihovny pro zaznamenání dynamických alokací a~předání parametrů.

\begin{minted}{bash}
$ LD_PRELOAD=${CUPR_BUILD_DIR}/runtimetracker/libruntimetracker.so \
    LD_LIBRARY_PATH=${CUPR_BUILD_DIR}/runtime \
    FORMAT=PROTOBUF \
    COMPRESS=1 \
    ./cuda
\end{minted}

Po ukončení programu se v~jeho pracovním adresáři vytvoří složka se zaznamenanými údaji. Ty lze poté zobrazit ve vizualizační aplikaci.

\subsubsection{Parametry}
\label{manual:parameters}
V~tabulce \ref{tab:parameters} jsou vypsány názvy, datové typy a~implicitní hodnoty parametrů, kterými lze ovlivnit instrumentaci a~běh profilovaných programů. Veškeré parametry se předávají pomocí proměnných prostředí. Pokud je v~tabulce uveden kontext \emph{překladač}, tak se parametr předává při překladu programu překladači Clang. Pokud je kontext \emph{program}, tak se parametr předává při spouštění instrumentovaného programu. U~parametrů typu \texttt{bool} se pravdivá hodnota zadává jako 1 a~nepravdivá hodnota jako 0.

\begin{table}[h]
	\centering
	\caption{Parametry profilovacího nástroje}
	\label{tab:parameters}
	\bgroup
	\def\arraystretch{1.2}
	\begin{tabularx}{\textwidth}{|c|c|c|c|Y|}
		\hline
		\textbf{Název} & \textbf{Typ} & \textbf{Implicitně} & \textbf{Kontext} & \textbf{Popis} \\
		\hline
		INSTRUMENT\_LOCALS & bool & vypnuto & překladač & Instrumentace lokálních přístupů \\
		\hline
		KERNEL\_REGEX & řetězec & & překladač & Regulární výraz pro filtrování kernelů \\
		\hline
		BUFFER\_SIZE & číslo & 1000000 & program & Velikost bufferu pro přístupy [počet přístupů] \\
		\hline
		PRETTIFY & bool & vypnuto & program & Odsazení pro lepší čitelnost JSONu \\
		\hline
		COMPRESS & bool & vypnuto & program & Gzip komprimace vygenerovaných souborů \\
		\hline
		FORMAT & řetězec & JSON & program & Serializační formát (PROTOBUF, CAPNP, JSON) \\
		\hline
		HOST\_MEMORY & bool & vypnuto & program & Použití paměti CPU pro ukládání přístupů \\
		\hline
		THREAD\_POOL & bool & zapnuto & program & Paralelizace formátování přístupů \\
		\hline
	\end{tabularx}
	\egroup
\end{table}

\subsubsection{Formát generovaných souborů}
Při instrumentaci a~spouštění instrumentovaných programů se generují tři typy souborů. Jejich specifikace ve formátu JSON Schema je v~kořenovém adresáři projektu ve složce \texttt{schema}. Soubor s~informacemi o~běhu celé aplikace má název \texttt{run.json} a~vytváří se při každém spuštění instrumentovaného programu. Dalším typem souboru je soubor s~metadaty, který je vždy pojmenován \texttt{<kernel>.metadata.json}, kde kernel je název kernelu, jehož metadata soubor obsahuje. Tento soubor je vytvořen pro každý instrumentovaný kernel při překladu aplikace. Posledním a~nejdůležitějším typem souboru je soubor s~paměťovými alokacemi. Ten se vytváří za běhu programu při každém spuštění instrumentovaného kernelu. Tento soubor by bylo komplikované popsat pomocí JSON Schema, jeho obsah si lze prohlédnout v~definici struktur pro Protobuf nebo Cap'n Proto. Ty se nacházejí ve složkách \texttt{collection/runtime/format/protobuf} a~\texttt{collection/runtime/format/capnp}.

\subsection{Webová aplikace}
Vizualizační aplikace se nachází ve složce \texttt{dashboard}, následující příkazy předpokládají spouštění z~této složky.

\subsubsection{Závislosti}
Pro instalaci a~spuštění vizualizační webové aplikace je nutné nainstalovat běhové prostředí Javascriptu \emph{NodeJS}\footnote{\url{https://nodejs.org}} alespoň ve verzi 8 a~jeho balíčkovací manažer \emph{npm}\footnote{\url{https://www.npmjs.com}}. Závislosti projektu lze nainstalovat tímto příkazem:
\begin{minted}{bash}
    $ npm install
\end{minted}

\subsubsection{Použití}
Aplikaci je poté možno spustit přímo ve vývojářském režimu:
\begin{minted}{bash}
    $ npm start
\end{minted}
nebo ji lze sestavit do statických HTML, CSS a~Javascript souborů a~poté použít libovolný webový server k~servírování aplikace ze složky \texttt{build}:
\begin{minted}{bash}
    $ npm run build
\end{minted}

Aplikace při spuštění ve vývojářském režimu běží na portu 3000, lze se na ni tedy dostat v~prohlížeči pomocí adresy \url{http://localhost:3000}.
Webovou aplikaci si lze prohlédnout také na této adrese: \url{https://kobzol.github.io/cuda-profile}.
